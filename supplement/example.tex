\documentclass{article}
\usepackage{statrep}
\begin{document}

This document is named \texttt{example.tex}. 
It contains the complete \textit{Getting Started}
example from the \textit{StatRep User's Guide}.
You can generate the final PDF document \texttt{example.pdf} as follows:

\begin{description}
\item[Generate the SAS program]\mbox{}\newline
Compile this document with \texttt{pdflatex}.
The \texttt{StatRep} package automatically generates a SAS program from
the document source. The program is named
\texttt{example\_SR.sas} and it is created in the current directory.

\item[Capture the SAS outputs]\mbox{}\newline
Run the SAS program \texttt{example\_SR.sas}. 
The SAS working directory must be the directory that contains this document.

\item[Create the final PDF]\mbox{}\newline
Compile this document with \texttt{pdflatex} once more. The outputs that SAS generated
in the preceding step are now included in the final PDF \texttt{example.pdf}.
\end{description}

The code in the \texttt{Datastep} environment is written
unchanged to the generated SAS program.

\begin{Datastep}
proc format;
   value $sex 'F' = 'Female' 'M' = 'Male';
data one;
   set sashelp.class;
   format sex $sex.;
run;
\end{Datastep}

  The code in the \texttt{Sascode} environment is parsed before it is
  written to the generated SAS program. For example, lines that begin with the string
  \texttt{\%*;} are written to the SAS program and are not displayed in the 
  final document. The other lines in this example are written to the program 
  and are displayed in the final document.
  
  The first line of the following code block can be seen only
  in the \LaTeX\ source file and in the generated SAS program. 
  The line insures that ODS Graphics are enabled.

\begin{Sascode}[store=class]
%*; ods graphics on;
proc reg;
    model weight = height age;
run;
\end{Sascode}

  The \texttt{Listing} and \texttt{Graphic} tags convey information to
  \LaTeX\ and to SAS. The tags specify the names of the output files to
  insert into the document and the captions for the output.
  Additionally, they specify the names of the output files to create
  and which ODS objects to capture.

\Listing[store=class,
         caption={Regression Analysis}]{rega}

\Graphic[store=class, 
         caption={Graphs for Regression Analysis}]{regb}

In this short example only the defaults are used. That is, all output
objects are selected and displayed.

\end{document}